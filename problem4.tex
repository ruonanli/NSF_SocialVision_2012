% !TEX root = SocialVision2012.tex

%\subsubsection{Noisy and incomplete network reconstruction}
\label{sec:reconstruct}

\boldstart{Reconstructing noise and missing data}. Even with identity recognition aside, the social information extracted from images and videos will be very noisy and incomplete. In many situations, images and videos will be of low quality; agents will exhibit significant pose variations and be occluded; tracking systems will become lost in clutter; and estimates of head pose, body pose, expression, etc. will be plagued by uncertainty. Consequently, missing and noisy links will be especially prevalent in visually-sensed social networks. As part of the proposed activity, we will build on successes in link prediction~\cite{Goldberg,Liben-Nowell,TaskarWAK03} and network completion~\cite{Clauset,Guimera,HannekeX09,KimL11}, by developing reconstruction tools that are better suited to highly-noisy and multi-view visually-sensed social networks.

To accommodate missing data, we will modify our representation for a social network graph of $K$ nodes by including a visibility matrix for each view. That is, $G\triangleq\{A^{(v)}, Q^{(v)}\} v=1,2,\cdots,V$, with $A^{(v)}$  the $K\times K$ affinity matrix for view $v$ and $Q^{(v)}$  the corresponding $K\times K$ visibility matrix for that view. If $Q^{(v)}(i,j)=1$, then $A^{(v)}(i,j)$ is the weight describing the tie or closeness between node $i$ and node $j$ estimated from the $v$th view; otherwise if $Q^{(v)}(i,j)=0$ then $A^{(v)}(i,j)$ is a missing number indicating the lack of information in this view. Our objective is to complete the missing links ($Q^{(v)}(i,j)=0$) by estimating the proper weights for these missing links.
%In the case that the views directly correspond to low-level visual cues, we may imagine that the ties between the pairs of members should not vary among different views due to different sensing modalities, and therefore there exist a unique community structure underlying all views. A primary objective in this case, is that how we may discover the community (clustering) effect from this partially observed multi-view network, together with filling the missing links with a proper weight. We refer to this primary task as network reconstruction.

The key observation is that structure across views can be used to transfer information from one to the other, effectively filling holes by borrowing information from other views. This can succeed as long as errors are incoherent across views, which is a reasonable expectation in practice. To operationalize this, we imagine that each node $i$ in the graph can be uniquely identified with a point $\vx_i$ in a Euclidean space of some dimension, and that the distance between each pair of points $(i,j)$ in this space can be interpreted as form of  view-invariant dissimilarity between the two nodes. Furthermore, we imagine that every view-specific affinity $A^{(v)}$ can be obtained by a simple global linear transform of the view-invariant Euclidean distance. This model is supported by the following theorem, based on results from multi-dimensional scaling~\cite{CoxMDS} (proof omitted due to space constraints), implying that any graph affinity can be analytically transformed to Euclidean distances between points.

\begin{quote}
\textbf{Theorem}. \textit{If $A$ is a symmetric affinity matrix with all zeros on the diagonal and positive numbers everywhere else, there exists a constant $c$ such that $(\frac{1}{A(i,j)^2}+c)^{\frac{1}{2}}$ is the Euclidean distance between point $i$ (representing node $i$) and point $j$ (representing node $j$) in an Euclidean space, where $c\geq\lambda$, the smallest eigenvalue of $\Lambda=H\Gamma H$, $H=\mathbf{I}-\frac{\mathbf{1}\mathbf{1}^T}{K}$, and $\Gamma(i,j)=-\frac{1}{2A(i,j)^2}$.} 
\end{quote}

The theorem guarantees that each node can be uniquely identified with a point in a Euclidean space, and then the Euclidean-embedded nodes $\vx_i$ and the multi-view network $G$ can be related by
\begin{equation}\label{eq:embed}
(\vx_i-\vx_j)^{T}\Sigma^{(v)}(\vx_i-\vx_j)=\left(\frac{1}{A^{(v)}(i,j)^2}+c^{(v)}\right)^{\frac{1}{2}}+\epsilon^{(v)}_{ij}, \forall Q^{(v)}(i,j)=1,
\end{equation}
where $\Sigma^{(v)}$ is a symmetric semi-positive definite matrix specific to the $v$th view, and $\epsilon$ is a residual. By doing so, different views are unified, and any application-dependent network priors or regularizations are straightforward to be characterized through the embedded points $\vx_i$. 

This model we propose to explore differs from existing analyses of embedding~\cite{Hoff01latentspace,Hancocklatent}, multi-view networks \cite{AiroldiBFX08,Kim12}, and network completion~\cite{Clauset,Guimera,HannekeX09,KimL11}, because it uses a latent Euclidean embedding to \emph{simultaneously} consider missing data and multiple views. It promises more effective use of structure that exists across distinct views in a network, something that we view as being very important to the success of vision-based network reconstruction. 

%A unified community clustering effect, as the social network prior, may consequently  be modeled in the Euclidean space as well, for example, via
%\begin{equation}\label{eq:kmeans}
%Z=\arg\min_{D,\hat{Z}}\|X-D\hat{Z}\|^{2}_{2}, \textup{s.t.} \hat{Z}^{T}\mathbf{1}=\mathbf{1},
% \end{equation}
%where $X=[\vx_1,\vx_2,\cdots,\vx_K]$, each column of $D$ represents the center of a cluster, and $Z, \hat{Z}$ are a binary matrices by which we assign a node to a cluster among the communities of interest given by $D$. Upon learning the overall model incorporating the essential components in (\ref{eq:embed})(\ref{eq:kmeans}). It is straightforward to reconstruct the noisy and the missing affinities through transforms of Euclidean distances, and to investigate the underlying community structure invariant of views.

%\subsubsection{Closing the loop}
\label{sec:closeloop}

\boldstart{Closing the loop}. A longer term goal of our research is to allow synergistic collaboration between visual recognition processes and social reconstructive processes. We have argued that visual recognition processes can provide significant information about an underlying social network, but the converse is also true. As shown by PI Zickler~\cite{Stone2008,Stone2010} and others, social network information can serve as valuable context to improve recognition. We imagine a future in which these two processes work together. When uncertainty in a face recognition system leads to low confidence in identities, the uncertainty propagates to the extracted social cues $\vy$ and therefore to the inferred social network graph $G$. However, by reconstructing and denoising the multi-view graph as proposed above, we can improve our estimate of the underlying social network, and then carry this information back to the image data, use the the improved social network as context to correct  errors and improve recognition models.
