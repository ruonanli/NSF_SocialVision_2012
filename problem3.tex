\subsection{Robust Reconstruction of Multiview Networks}
\label{sec:reconstruct}

In the previous two sections, we presented two core problems on socially-assisted vision: Target recognition using social contexts and socially-meaningful behavior parsing. In this and the next sections, we propose two tasks regarding network sensing from visual information.

The most fundamental problem in social network may be probably how to represent such a network, and only with a good representation one is able to perform learning and inference in a practical manner. A weighted graph of $K$ nodes, with a non-negative weight associated with each edge between a pair of nodes, has been dominantly used to represent a community of $K$ members, where the weight depicts the closeness, or ties, between the two member. Despite the effectiveness and efficiency of such a representation, we envisage and have justified in the introduction that such models can be significantly enriched by introducing visual sensors, and we argue that a more appropriate network representation should account for realistic conditions arising from not only visual but also other conventional information sources. We propose in this section a new framework by which we re-represent a visually-assisted sensed social network in a realistic world.

First, the description of a network is generally in multiple views from visual sensors and even multiple types of conventional sensors, as we have mentioned in Section \ref{sec:recognition}, where a view refers to a specific type of attribute, computed by a particular computer vision algorithm, that describes the closeness or ties between two members. The social relaionship between two faces(humans) in Facebook, for example, can be in three views: The number of their shared friends, the number of their co-occurrence in pictures, and their relative poses in images. Second, the social networks are usually partially observed, meaning that not all these views are `visible'. Some links, i.e., edges in the graph cannot be evaluated with a weight, for example, because of the incapability of the vision function on a particular image. Third, a network almost always embeds community structures, meaning that the members belongs to different clusters, within each of which the members share close ties with each other, and members from different clusters are different from each other. Last but not least, for the same group of members there are usually multiple types of community structures and the communities overlap with each other. To explain this, consider a network of members who are mutually each others family members, workmates, or friends. Apparently, the clustering patterns for the families, for the career, and for friendship are all different, while each member can be fully situated in each and every of these types of communities.


A practically meaningful representation of social network must accommodate all the opportunities and challenges brought by these practical conditions in a realistic network society. To begin with, let us assume for the moment that a single rather than multiple overlapping community structure. Formally, we can represent the undirected weighted graph of $K$ nodes as $G$ describing the connections of $K$ members, where $G\triangleq\{A^{(v)}, Q^{(v)}\}, v=1,2,\cdots,V$, $A^{(v)}$ is the $K\times K$ affinity matrix describing the ties or closeness for the $v$th view, and $Q^{(v)}$ is the $K\times K$ visibility matrix for that view. If $Q^{(v)}(i,j)=1$, then $A^{(v)}(i,j)$ is the weight describing the tie or closeness between node $i$ and node $j$ estimated from the $v$th view; otherwise if $Q^{(v)}(i,j)=0$ then $A^{(v)}(i,j)$ is a missing number indicating the lack of information of this link in this view. A primary objective, is that how we may discover the community (clustering) effect from this partially observed multi-view network, together with filling the missing links with a proper weight. We refer to this primary task as network reconstruction.

As the ties between the pairs of members should not vary among different views due to different sensing modalities, a desirable way to represent the group of $K$ nodes is to find a `view-invariant' representation for them. To this end, we imagine that if each node $i$ in the graph can be uniquely identified with a point $\vx_i$ in an Euclidean space, then the distance between each pair of points $(i,j)$ can be regarded as the `view-invariant' dissimilarity between the two nodes, and the view-specific affinity $A^{(v)}(i,j)$ may be derived from certain view-specific transform of the Euclidean distance between the two points $i$ and $j$. Moreover, the community structure is also easier to be characterized as clustering effect in the Euclidean space, and the observability of the affinities is also straightforward to present through the indicating variable $Q^{(v)}(i,j)$. The question whether all these conjectures can be successful now boils down to the question whether there exists a universal connection between Euclidean distances and arbitrary graph affinities.

The answer is YES. We have proved the following theorem, which implies that arbitrary graph affinities may always be analytically transformed into Euclidean distances between points. The proof of the theorem is omitted due to space limitation.

\vspace{5pt}
\textbf{Theorem}. \textit{If $A$ is symmetric affinity matrix with all zeros on the diagonal and positive numbers everywhere else, then there exists a constant $c$ such that $(\frac{1}{A(i,j)^2}+c)^{\frac{1}{2}}$ is the Euclidean distance between point $i$ (representing node $i$) and point $j$ (representing node $j$) in an Euclidean space, where $c\geq\lambda$, the smallest eigenvalue of $\Lambda=H\Gamma H$, $H=\mathbf{I}-\frac{\mathbf{1}\mathbf{1}^T}{K}$, and $\Gamma(i,j)=-\frac{1}{2A(i,j)^2}$.} $\blacksquare$
\vspace{5pt}


The above theorem provides theoretical guarantee that each member (node) can be uniquely identified with a point in a Euclidean space, because now the Euclidean embedded nodes $\vx_i$'s and the generic social network representation $G$ are naturally related as
\begin{equation}\label{eq:embed}
d^{(v)}(\vx_i, \vx_j, \theta^{(v)})=(\frac{1}{A^{(v)}(i,j)^2}+c^{(v)})^{\frac{1}{2}}, \forall Q^{(v)}(i,j)=1,
\end{equation}
where $d^{(v)}(\cdot, \cdot, \theta^{(v)})$ is a properly defined distance in the Euclidean space specific to the $v$th view with parameter $\theta^{(v)}$. Meanwhile, the community structure, consequently, may be modeled in the Euclidean space as well, for example, via
\begin{equation}\label{eq:kmeans}
Z=\arg\min_{D,\hat{Z}}\|X-D\hat{Z}\|^{2}_{2}, \textup{s.t.} \hat{Z}^{T}\mathbf{1}=\mathbf{1},
 \end{equation}
where $X=[\vx_1,\vx_2,\cdots,\vx_K]$, each column of $D$ represents the center of a community, and $Z, \hat{Z}$ are a binary matrices by which we assign a node to a community among the communities of interest given by $D$. Upon learning the overall model incorporating the essential components in (\ref{eq:embed})(\ref{eq:kmeans}). It is straightforward to reconstruct the missing affinities through transforms of Euclidean distances, and to investigate the underlying community structure invariant of views.

On our research agenda, we will also extend this basic model into one that accommodates multiple overlapping communities.