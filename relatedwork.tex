\Section{Background and Related Work}

In order for computer vision to be `socially-aware', we require concepts, perspectives, semantics, tools, and models to be shared and interrelated among diverse areas from both computer vision and social networks. Research in these areas has been progressing over the past decades, and can be roughly divided into five focused topics: Qualitative and quantitative sociology, face analysis and recognition, human behavior recognition, socially-contextual visual analysis, as well as computational network models. This section reviews related work in these sub areas. Note that space restrictions prohibit thorough and comprehensive discussions on any of the five subareas. Instead, we restrict our attention to the work most related to our proposed research, and refer the reader to survey articles.This background provides a foundation for the proposed work in Sect.~\ref{sec:proposed-research}.

\boldstart{Face Analysis and Recognition}. Face recognition is a mature topic in computer vision. Before a computer vision system can recognize \cite{Chellappa:face} faces, it must usually first detect \cite{ViolaJones,Zhang:detect} them, track \cite{Comaniciu:track} them, and align \cite{Matthews:AAM,Lucey:AAM,Mumford:face} them if necessary. Social media has spurred renewed interest in face recognition: There is current interest in images captured ``in the wild'' as opposed to those from controlled face databases. Moderately large labeled face databases have been collected by exploiting captions associated with television and video~\cite{berg2004naf,berg2005sp,Everingham06a,huang:lfw}. Based on this and other data, we have seen improving schemes for alignment~\cite{huang:lfw,HolubMoreelsPeronaFG08}, clustering~\cite{HolubMoreelsPeronaFG08}, biologically-inspired descriptors \cite{PintoZickler2011}, classifying faces as being ``same or not-same"~\cite{nowak2007lvs}, and recognizing attributes such as mustaches and eye wear~\cite{LNCS53050340}. While these results are encouraging, face recognition is by no means a solved problem. Current best performance of  ``same/not-same" classifiers on the LFW database~\cite{huang:lfw}, for example, yields a mean classification accuracy of only 79\% \cite{Pinto2009}. This is not surprising because recognition based on faces alone is fundamentally limited. Dramatic improvements can be achieved, however, by combining these systems with useful social contexts as we seek to do here.

The recognized faces exhibit more information than those regarding identities and attributes, among which many are tightly related to the social semantics. Facial expression analysis from still images and video sequences has been under close attention of computer vision as well \cite{Yacoob:expression,delaTorre:expression,Essa:expression}, though the study has not been put under social contexts. Interactive behaviors are strongly informed by gazes \cite{Hanson} and head poses \cite{Murphy-Chutorian:pose}, and the latter are meanwhile modulated by the inter-person relations during an interactive activity.


\boldstart{Human Behavior Recognition}. In addition to facial behaviors, social clues are abundantly conveyed by body language of humans. Human behaviors are not only in the form of moderate body motion - gesture \cite{Mitra:gesture}, but also in the form of articulations - full-body action \cite{Ryoo:action,Poppe}. The description of these human behaviors builds upon low-level spatio-temporal descriptors \cite{Dalal:HOG,Dollar:STIP,Laptev:STIP,Brox:flow}, and is hierarchically organized into high-level grammar structure \cite{Niebles2007,Niebles2006} for recognition.  Related tasks also include detecting motion from clutter \cite{Li2010}, comparing two behaviors \cite{LiPAMI 2012}, as well as adapting behavior representations between different modalities \cite{LiZickler2012,Li2011}, which are all among PI Li's expertise.

Studies on social behavior involving more than one individual had not begun until efforts were made in videos of small-scale indoor meeting \cite{GaticaPerez,McCowan:meeting}, where the functionalities of tracking \cite{Smith:track} and pose estimation \cite{Ba:meeting} are integrated in dynamic hidden Markov models (HMM) \cite{Zhang:meeting}. More efforts emerged in the past few years, spanning pairwise interactive activities \cite{UTdata}, as well as group-wise, or collective activities \cite{Choi:context,Choi:recogtrack,Amer:group,Lan:Group}, as well as PI Li's recent studies on interactions in sport games \cite{LiIJCV2012}.  Despite of an increasing interest in the group-wise activities, a gap between the existing work and our proposed research is obvious: Instead of treating the group of humans as a unified and `intact' target, or sample for categorization, we aim to fill in the blank area of overlaying the group of individuals with the social network among them, as to be detailed in Sect.~\ref{sec:proposed-research}.


\boldstart{Qualitative and Quantitative Sociology}.  Despite of the lack of visual sensor, social network has been among the most persistent research interests for years. Sociology, where the modern social network research is rooted, can be traced back to decades ago when investigations are performed on human beings' elementary social properties \cite{Darwin,Thomkins,Goffman,Kendon1990,Ekman,Hoyle,Tannen}. Interactions and relations among socialized individuals have been extensively under the study of sociology in either a qualitative or a quantitative manner \cite{Goodwin2000,Goldin,Goodwin2007,Kendon2010}, while the arrival of the era of contemporary communication and internet brings the human society into a real technically interconnected community, so that analytical and statistical approaches can be widely applied to new socializing modalities, such as emails \cite{Eckmann} and mobile phones calls \cite{Onnela,Eagle}. Rich and heterogeneously structured communities can be abstracted from various domains, including production and consumption \cite{Watts}, transportation and travel \cite{Gonzalez},  as well as politics \cite{Iacus}. All together has shaped the interdisciplinary study on the networked life of human beings, namely `computational social science', into an infancy \cite{Lazer2009}. As has been discussed, the perspectives and problems in the computational social science, will infuse the computer vision with nutritional ingredients, and eventually benefit from the outcome of such research.

\boldstart{Computational Network Models}. Among the overwhelming volumes of literature on social networks in computer science, machine learning, and statistics, our research will be more closely tied to several topics, including graph cut \cite{Boykov:segmentation} together with its application to clustering and community detection \cite{Ng:spectral,Filippone:clustering}, general problems of clustering \cite{Xu:clustering}, as well as graph matching \cite{West:Graph,Caetano:graph}. Probabilistic mechanisms for depicting the dynamics of a small-scale community, especially in conversational scenes, are available as well \cite{Basu:meeting,Dong}.among which a variation of models can be employed , including coupling model among agents \cite{Brand:CHMM}, mixed memory model \cite{Choudhury:MHMM}, and influence model \cite{Pan:influence}.

Statistical network models, as well as a diversity of problems regarding learning and inference on graphs, are particularly useful in unifying visual representations and relational representations. It becomes difficult even to enumerate a few among the broad literature arising from prosperously studies in machine learning community and statistics.  We refer the reader to \cite{Kolacyzk} for a comprehensive coverage of representative work on statistical analysis of networks, and \cite{Goldenberg} for popular models widely used by researchers. An overview of these statistical models from a sociology perspective can be found in \cite{Snijders}, and a most recent survey on various aspects in the framework of learning on graphs or relational structures is available in \cite{Rossi}.


\boldstart{Socially-Contextual Visual Analysis}. Objects usually do not appear independently in photographs, and certain object categories are more likely to appear together than others. This is especially true if the object categories being considered are people. Preliminary attempts have been appearing, though sparsely, in the past decade in order to exploit this intuitive but fundamental fact. 

The performance of identity recognition has been shown to improve from exploiting context information such as the clothing that people wear over the course of a few hours or a day~\cite{anguelov2007cir, zhang2003aah,  song2006cah, sivic2006fpr}. Similarly, it is possible to use additional metadata such as the temporal and spatial context of photographs~\cite{naaman2005lcr, zhao2006apa}, and the frequency with which certain pairs of people appear together in a personal photo collection~\cite{anguelov2007cir}. More recent efforts have been devoted to reveal the basic social attributes associated with the individuals in the photo albums \cite{Gallagher,Wang2010}. Temporal information conveys even more cues for us to unveil the inter-relational semantics among individuals within a community. Social roles of the characters in movies can be inferred from visual analysis \cite{Ding2010,Ding2011}, and organizational roles of employees in a workspace can also be robustly agglomerated from long-term surveillance \cite{Yu2009,Zhang2011}.

Research group led by PI Zickler has made the pilot contribution in face recognition using social contexts in the internet era \cite{Stone2008,Stone2010}. Drawing upon the newly-popular phenomenon of "tagging," we construct some of the first face identification datasets that are intended to model the digital social spheres of online social network members, and we examine various qualitative and quantitative properties of these image sets. The identification datasets we present here include up to 100 individuals, making them comparable to the average size of members' networks of "friends" on a popular online social network, and each individual is represented by up to 100 face samples that feature significant real-world variation in appearance, expression, and pose. We illustrate that the machine-readable "social context" in which shared photos are often embedded can be applied under standard statistical formalism to boost face identification accuracy. These results are in companion to more recent results \cite{Dikmen:classify,LeeBMVC2011,Murillo2012,Poppe2012}, and has spurred the extension of the approach to large-scale joint classification of a image set \cite{McAuley:socialclassify}. All these preliminary work, namely `social signal processing' \cite{Pantic}, provide foundations, insights, as well as inspirations for us to develop a systematic socially-aware visual system, as we propose in this research.
