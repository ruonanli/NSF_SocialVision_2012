% !TEX root = SocialVision2012.tex
\Section{Background and Related Work}

We will build on knowledge and tools from computer vision, sociology, and social network analysis; and this section discusses the relevant background in five parts. Space restrictions prohibit a comprehensive discussion of these five subareas, so we only describe the work most related to our proposal and refer the reader to appropriate existing surveys. The background in this section provides a foundation for our proposed research in Sect.~\ref{sec:proposed-research}.

\boldstart{Facial analysis and identity recognition}. One set of technologies we will leverage to infer social network information from images and videos is that for detecting faces; tracking them over time; and analyzing them to extract information about pose, expression, identity, gender, age, ethnicity, and so on.  Current computer vision systems are  far from perfect on these tasks, but their performance is now at useful levels thanks to the explosion of personal and community photo collections, the introduction of social tagging~\cite{Stone2008,Stone2010} and other (semi-)automatic tagging mechanisms~\cite{berg2004naf,berg2005sp,Everingham06a,huang:lfw}\todd{Add to list paper by Ramanan and Baker about the ``Friends" dataset}, and public demand for applications enabled by facial analysis. There now exist practical systems that automatically provide from images and videos useful information about facial location (e.g.,~\cite{ViolaJones,Zhang:detect,Comaniciu:track}), gaze direction (e.g.,~\cite{Hanson}), head pose and expression (e.g.,~\cite{Murphy-Chutorian:pose,Matthews:AAM,Lucey:AAM,Mumford:face,Yacoob:expression,delaTorre:expression,Essa:expression,huang:lfw,HolubMoreelsPeronaFG08}), identity (e.g.,~\cite{PintoZickler2011})\todd{fix this citation}, and other attributes like gender, age, and ethnicity~\cite{LNCS53050340}. As  an example, a recent test by PI Zickler and colleagues on static images showed that a relatively simple computational pipeline is able to achieve close to 90\% accuracy on a 100-way identity recognition task when sixty or more annotated facial samples are available for each of the one-hundred individuals in the gallery~\cite{PintoZickler2011}. 

We will leverage these technologies in two distinct ways. First, we will use estimates of facial location, pose, and expression (when possible) as sources of information about the social interactions that occur in still images and videos. Second, we will use estimates of identity to associate these interactions with individuals (``nodes'') in the underlying social network. As stated in the introduction, one of the main challenges that we will address  in the proposed research is how to deal with the uncertainty that is inherent to these sources of information. 

\boldstart{Body analysis and individual activity recognition}. We will also leverage existing mature technology for detecting individual bodies (or visible body parts)~\cite{Dalal:HOG}\todd{add poselets}, tracking them over time~\cite{}\todd{add ``Tracking in a Dense Crowd Using Multiple Cameras'' (IJCV 2010) and other citations if you have them}, and analyzing them to extract information about positions, velocities, pose, gestures, and other fine-scale movements~\cite{Mitra:gesture,Ryoo:action,Poppe}. As is the case for facial analysis, practical techniques exist for a variety of image quality levels, from high-resolution spatio-temporal descriptors~\cite{Dollar:STIP,Laptev:STIP,Brox:flow} and activity grammars~\cite{Niebles2007,Niebles2006} to very coarse histograms-of-flow for body motion in low-resolution video~\cite{}\todd{add citation to Efros action recognition from coarse flow (and others) here}. These descriptors have proven useful for detecting pre-defined action categories in the presence of clutter (e.g.,~\cite{Li2010} by PI Li), and for discriminating between different categories of individual actions (e.g., walking, running, hand waving)~\cite{}\todd{KTH dataset results. Others?} even when there are substantial changes in camera position (e.g.,~\cite{LiZickler2012} by PIs Li and Zickler). Our goal is to use similar descriptors for a different purpose: detecting and recognizing interactions that involve \emph{groups} of individuals. 
%Put this somewhere else:  Related tasks also include detecting motion from clutter \cite{Li2010}, comparing two behaviors \cite{LiPAMI 2012}, as well as adapting behavior representations between different modalities \cite{LiZickler2012,Li2011}, which are all among co-PI Li's expertises.

\boldstart{Group activity analysis}. In computer vision, analyses of behaviors involving more than one person are less common. The earliest efforts were devoted to analyzing indoor meetings involving handfuls of participants \cite{GaticaPerez,McCowan:meeting}, where descriptors derived from tracking \cite{Smith:track} and pose estimation \cite{Ba:meeting} have been integrated in dynamic hidden Markov models (HMM)~\cite{Zhang:meeting}\todd{need more here. What was the output of their system? How is this different or similar to us? Has any work on meetings defined interaction categories, or recovered social network information from occurrences of these categories?}. Recent efforts have explored more complex visual environments, including the detection and recognition of a small number of pre-defined two-person interaction categories (e.g., hand-shaking, pushing, hugging)~\cite{UTdata}\todd{More here. Cite the related previous work in the CVPR submission.}; the detection of group conversations in a cocktail-party scenario~\cite{Cristani:fformation}; the detection of certain pre-defined categories of collective large-group behavior~\cite{Choi:context,Choi:recogtrack,Amer:group,Lan:Group}; and PI Li's work on segmenting offensive and defensive players in sports matches~\cite{LiIJCV2012}. We aim to build on the promising results of these early and disparate approaches but launching a large scale research effort toward a general framework for analyzing social interactions in all of these environments and more. We also seek to go one step further by extracting from these observed interactions information about the underlying social network.

\boldstart{Qualitative and quantitative sociology}.  Despite of the lack of visual sensor, social network has been among the most persistent research interests for years. Sociology, where the modern social network research is rooted, can be traced back to decades ago when investigations are performed on human beings' elementary social properties \cite{Darwin,Thomkins,Goffman,Kendon1990,Ekman,Hoyle,Tannen}. Interactions and relations among socialized individuals have been extensively under the study of sociology in either a qualitative or a quantitative manner \cite{Goodwin2000,Goldin,Goodwin2007,Kendon2010}, while the arrival of the era of contemporary communication and internet brings the human society into a real technically interconnected community, so that analytical and statistical approaches can be widely applied to new socializing modalities, such as emails \cite{Eckmann} and mobile phones calls \cite{Onnela,Eagle}. Rich and heterogeneously structured communities can be abstracted from various domains, including production and consumption \cite{Watts}, transportation and travel \cite{Gonzalez},  as well as politics \cite{Iacus}. All together has shaped the interdisciplinary study on the networked life of human beings, namely `computational social science', into an infancy \cite{Lazer2009}. As has been discussed, the perspectives and problems in the computational social science, will infuse the computer vision with nutritional ingredients, and eventually benefit from the outcome of such research.

\boldstart{Computational network models}. Among the overwhelming volumes of literature on social networks in computer science, machine learning, and statistics, our research will be more closely tied to several topics, including graph cut \cite{Boykov:segmentation} together with its application to clustering and community detection \cite{Ng:spectral,Filippone:clustering}, general problems of clustering \cite{Xu:clustering}, as well as graph matching \cite{West:Graph,Caetano:graph}. Probabilistic mechanisms for depicting the dynamics of a small-scale community, especially in conversational scenes, are available as well \cite{Basu:meeting,Dong}.among which a variation of models can be employed , including coupling model among agents \cite{Brand:CHMM}, mixed memory model \cite{Choudhury:MHMM}, and influence model \cite{Pan:influence}.

Statistical network models, as well as a diversity of problems regarding learning and inference on graphs, are particularly useful in unifying visual representations and relational representations. It becomes difficult even to enumerate a few among the broad literature arising from prosperously studies in machine learning community and statistics.  We refer the reader to \cite{Kolacyzk} for a comprehensive coverage of representative work on statistical analysis of networks, and \cite{Goldenberg} for popular models widely used by researchers. An overview of these statistical models from a sociology perspective can be found in \cite{Snijders}, and a most recent survey on various aspects in the framework of learning on graphs or relational structures is available in \cite{Rossi}.


\boldstart{Enhancing recognition with social context}. Objects usually do not appear independently in photographs, and certain object categories are more likely to appear together than others. This is especially true if the object categories being considered are people. Preliminary attempts have been appearing, though sparsely, in the past decade in order to exploit this intuitive but fundamental fact. 

The performance of identity recognition has been shown to improve from exploiting context information such as the clothing that people wear over the course of a few hours or a day~\cite{anguelov2007cir, zhang2003aah,  song2006cah, sivic2006fpr}. Similarly, it is possible to use additional metadata such as the temporal and spatial context of photographs~\cite{naaman2005lcr, zhao2006apa}, and the frequency with which certain pairs of people appear together in a personal photo collection~\cite{anguelov2007cir}. More recent efforts have been devoted to reveal the basic social attributes associated with the individuals in the photo albums \cite{Gallagher,Wang2010}. Temporal information conveys even more cues for us to unveil the inter-relational semantics among individuals within a community. Social roles of the characters in movies can be inferred from visual analysis \cite{Ding2010,Ding2011}, and organizational roles of employees in a workspace can also be robustly agglomerated from long-term surveillance \cite{Yu2009,Zhang2011}.

Research group led by PI Zickler has made the pilot contribution in face recognition using social contexts in the internet era \cite{Stone2008,Stone2010}. Drawing upon the newly-popular phenomenon of "tagging," we construct some of the first face identification datasets that are intended to model the digital social spheres of online social network members, and we examine various qualitative and quantitative properties of these image sets. The identification datasets we present here include up to 100 individuals, making them comparable to the average size of members' networks of "friends" on a popular online social network, and each individual is represented by up to 100 face samples that feature significant real-world variation in appearance, expression, and pose. We illustrate that the machine-readable "social context" in which shared photos are often embedded can be applied under standard statistical formalism to boost face identification accuracy. These results are in companion to more recent results \cite{Dikmen:classify,LeeBMVC2011,Murillo2012,Poppe2012}, and has spurred the extension of the approach to large-scale joint classification of a image set \cite{McAuley:socialclassify}. All these preliminary work, namely `social signal processing' \cite{Pantic}, provide foundations, insights, as well as inspirations for us to develop a systematic socially-aware visual system, as we propose in this research.
