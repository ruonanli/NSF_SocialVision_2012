% !TEX root = SocialVision2012.tex
\Section{Background and Related Work}

We will build on knowledge and tools from computer vision, sociology, and social network analysis. This section discusses the relevant background in each, as well as some early explorations of their intersection. Space restrictions prohibit a comprehensive discussion of these five subareas, so we only describe the work most related to our proposal and refer the reader to appropriate existing surveys. The background in this section provides a foundation for our proposed research in Sect.~\ref{sec:proposed-research}.

\subsection{Existing tools from computer vision}

\boldstart{Facial analysis and identity recognition}. One set of technologies we will leverage to infer social network information from images and videos is that for detecting faces; tracking them over time; and analyzing them to extract information about pose, expression, identity, gender, age, ethnicity, and so on.  Current computer vision systems are  far from perfect on these tasks, but their performance is now at useful levels thanks to the explosion of personal and community photo collections, the introduction of social tagging~\cite{Stone2008,Stone2010} and other (semi-)automatic tagging mechanisms~\cite{berg2004naf,berg2005sp,Everingham06a,huang:lfw,YangBKR12}, and public demand for applications enabled by facial analysis. There now exist practical systems that automatically provide from images and videos useful information about facial location (e.g.,~\cite{ViolaJones,Zhang:detect,Comaniciu:track}), gaze direction (e.g.,~\cite{Hanson}), head pose and expression (e.g.,~\cite{Murphy-Chutorian:pose,Matthews:AAM,Lucey:AAM,Mumford:face,Yacoob:expression,delaTorre:expression,Essa:expression,huang:lfw,HolubMoreelsPeronaFG08}), identity (e.g.,~\cite{Chellappa:face}), and other attributes like gender, age, and ethnicity~\cite{LNCS53050340}. As  an example, a recent test by PI Zickler and colleagues on static images showed that a relatively simple computational pipeline is able to achieve close to 90\% accuracy on a 100-way identity recognition task when sixty or more annotated facial samples are available for each of the one-hundred individuals in the gallery~\cite{PintoZickler2011}. 

We will leverage these technologies in two distinct ways. First, we will use estimates of facial location, pose, and expression (when possible) as sources of information about the social interactions that occur in still images and videos. Second, we will use estimates of identity to associate these interactions with individuals (``nodes'') in the underlying social network. As stated in the introduction, one of the main challenges that we will address  in the proposed research is how to deal with the uncertainty that is inherent to these sources of information. 


%%%%%%%%%%%%%%%%%%%%%%%%%%%%%%%%%%%%%%%%%


\boldstart{Body analysis and individual activity recognition}. We will also leverage existing mature technology for detecting individual bodies (or visible body parts)~\cite{Dalal:HOG,poselet,pose_part}, tracking them over time~\cite{RamananFZ07,EshelM10}, and analyzing them to extract information about positions, velocities, pose, gestures, and other fine-scale movements~\cite{Mitra:gesture,Ryoo:action,Poppe}. As is the case for facial analysis, practical techniques exist for a variety of image quality levels, from high-resolution spatio-temporal descriptors~\cite{Dollar:STIP,Laptev:STIP,Brox:flow} and activity grammars~\cite{Niebles2007,Niebles2006} to very coarse histograms-of-flow for body motion in low-resolution video~\cite{EfrosBMM03}. These descriptors have proven useful for detecting pre-defined action categories in the presence of clutter (e.g.,~\cite{Li2010} by PI Li), and for discriminating between different categories of individual actions (e.g., walking, running, hand waving)~\cite{Weizmann,KTH} even when there are substantial changes in camera position (e.g.,\cite{Weinland:invariance2}, and ~\cite{LiZickler2012} by PIs Li and Zickler). Our goal is to use similar descriptors as a starting point of a different problem: detecting and recognizing social interactions that involve \emph{groups} of individuals. 

%Put this somewhere else:  Related tasks also include detecting motion from clutter \cite{Li2010}, comparing two behaviors \cite{LiPAMI 2012}, as well as adapting behavior representations between different modalities \cite{LiZickler2012,Li2011}, which are all among co-PI Li's expertises.


%%%%%%%%%%%%%%%%%%%%%%%%%%%%%%%%%%%%%%%%%

\subsection{Existing tools from sociology and network analysis}

\boldstart{Qualitative and quantitative sociology}.  Despite of the lack of visual sensor, social network has been among the most persistent research interests. Sociology, where the modern social network research is rooted, can be traced back to decades ago when investigations are performed on human beings' elementary social properties \cite{Darwin,Thomkins,Goffman,Kendon1990,Ekman,Hoyle,Tannen}. Interactions and relations among socialized individuals have been extensively under study in either a qualitative or a quantitative manner \cite{Goodwin2000,Goldin,Goodwin2007,Kendon2010}, while the arrival of the era of contemporary communication and internet brings the human society into a real technically interconnected community, so that analytical and statistical approaches can be widely applied to new socializing modalities such as emails \cite{Eckmann} and mobile phones calls \cite{Onnela,Eagle}. Rich and heterogeneously structured communities can be abstracted from various domains, including education \cite{Scherr2009}, production \cite{Watts}, transportation \cite{Gonzalez},  as well as politics \cite{Iacus}. All together has shaped the interdisciplinary study on the networked life of human beings, namely `computational social science' \cite{Lazer2009} and `social signal processing' \cite{Pantic}. These research will benefit socially-aware computer vision, for example, by providing socially-meaningful behavioral concepts to exploit, such as `f-formations' \cite{Kendon1990} which describes the spatial proximity among participants involved in a short-term interaction. Meanwhile, our proposed research on social interaction detection and discovery will allow learning of interaction categories in many new environments and for large group sizes over long periods, which is usually prohibitive for manual work as has been adopting in current sociological studies.


%%%%%%%%%%%%%%%%%%%%%%%%%%%%%%%%%%%%%%%%%

\boldstart{Computational network models}. Probabilistic mechanisms for depicting the dynamics of a small-scale group, especially in conversational scenes, are available in \cite{Basu:meeting,Dong}, with a variation of models including coupling model among agents \cite{Brand:CHMM}, mixed memory model \cite{Choudhury:MHMM}, and influence model \cite{Pan:influence}. Among the literature on social networks in computer science, machine learning, and statistics, our research will be closely tied to graph cut and clustering \cite{Ng:spectral,Boykov:segmentation,Filippone:clustering,Xu:clustering} as well as graph matching \cite{West:Graph,Caetano:graph}. Statistical network models, as well as a diversity of problems regarding learning and inference on graphs, are particularly useful in unifying visual representations and relational representations, in modeling a broad body of network phenomena (e.g., message propagation and small-world effect) and solving a diversity of network problems (e.g. link prediction and page recommendation). We refer the reader to \cite{Goldenberg,Kolacyzk} for a comprehensive coverage of representative work on statistical models of networks, together with perspectives from sociology \cite{Snijders} and recent progress on learning on relational structures \cite{Rossi}. In particular, research has considered the `multi-view' network representation, where a view may refer to a type of low-level observation or source that characterizes the affinities between nodes, such as citations between papers or the similarity of the words/topics in two papers \cite{ChangB09,WangMM05}, and a view can also refer to a type of high-level membership \cite{AiroldiBFX08,Kim12}. Missing links or nodes are common for realistic networks as accounted for by the task of network completion \cite{Clauset,Guimera,HannekeX09,KimL11}, and new links will be emerging over time as studied by link prediction \cite{Goldberg,Liben-Nowell,TaskarWAK03}. As to be elaborated, a visually sensed social network must consider these conditions and challenges simultaneously, instead of independently as previous work. We are also confronted with the new challenge of inferring high-level relational semantics from high-dimensional multi-modal image features. In addition, new problems arise in visually sensing of a social network and spur improved approaches for handling missing/noisy high-dimensional data. These will be our proposed contribution in Section \ref{sec:vis2net}.


%%%%%%%%%%%%%%%%%%%%%%%%%%%%%%%%%%%%%%%%%

\subsection{Recent Influences}

While computer vision and social network analysis have for the most past been investigated separately, there are some notable predecessors that have successfully united them for certain tasks in certain domains.

\boldstart{Enhancing recognition with social context}. With the growth of online community photo collections, a new sub-field is emerging that aims to use social network information as context to improve image-based identity recognition and scene understanding. For the most part, these methods rely explicitly on social network information extracted from ``Facebook friendships'' and other Internet-based relationship information, and they do not attempt to recover social information from the images themselves. Research group led by PI Zickler has made the pilot contribution in face recognition using social contexts in the internet era \cite{Stone2008,Stone2010}. In companion, similar attempts include exploiting context information such as the clothing that people wear over the course of a few hours or a day~\cite{anguelov2007cir, zhang2003aah,  song2006cah, sivic2006fpr}, or using additional metadata such as the temporal and spatial context of photographs~\cite{naaman2005lcr, zhao2006apa} and the frequency with which certain pairs of people appear together in a personal photo collection~\cite{anguelov2007cir}. These results are followed more recent results \cite{Dikmen:classify,LeeBMVC2011,Poppe2012}, and has spurred the extension of the approach to large-scale joint classification of a image set \cite{McAuley:socialclassify}. These methods solve a problem that is opposite to ours in that they use social network information from non-image sources to improve computer vision. We seek to do the opposite by using computer vision to infer social network information front he images themselves, with the belief that this information will complement the social information available from other non-image sources. They also provide an example of an application of our methods: Once images are used to recover social network information using our methods, this social network information can be used in turn to improve computer vision. This feedback also suggests unsupervised approaches that simultaneously optimized visual and social analyses. This is one of our long-term goals (see Sect.~\ref{sec:unsuperivsed}).

%%%%%%%%%%%%%%%%%%%%%%%%%%%%%%%%%%%%%%%%%

\boldstart{Inferring elementary social roles}. Seminal work has only appeared since the past three years, when researchers attempted to infer elementary social attributes such as `parents-children' among the detected targets, from simple geometric configurations and appearance descriptors of detected faces or body parts\cite{Gallagher,Wang2010,Murillo2012}, without connecting the image targets to members in the network. `Alliance' clusters over the characters in movies can be inferred from co-occurrences and shot anslysis \cite{Ding2010,Ding2011}, and the leader of a group of employees in a workspace can also be agglomerated from long-term surveillance \cite{Yu2009,Zhang2011}. These preliminary work will become degraded special cases of our proposed framework, which aims to explicitly mapping targets in larger image/video collection into nodes in the network of much larger scale. Instead of leveraging a few low-level visual cues for inferring against a small concept dictionary, our research will extract socially informative mid-level representations - interactions, and develop systematic regression machines from multi-type noisy visual cues to multi-view affinities.


\boldstart{Group activity analysis}. More related to our work are the small number of predecessors in computer vision for analyzing behaviors that involve more than one person. Many of the work recognize a group activity in an input video which is temporally cropped around the group activity and where all detected/tracked humans are participants of the underlying activity \cite{Intille:act,Ni:group,Lan:Group}, while some consider a salient group activity to span only a short duration of a long video \cite{Hongeng:act,Gong:act,Hakeem:act,Ba:meeting,McCowan:meeting,Choi:recogtrack,Vlad:group, Ryoo:group}. PI Li's work on segmenting offensive and defensive players in sports matches \cite{LiIJCV2012,Li2010} are the first set of work to put salient group activities in clutter of by-standers or distractors and accomplishes the task of spatial detection of group activities. However, the most generic case that we propose to address in this research, is instantaneous socially-informative interactive activities (rather than ) both embedded in clutter and occur within a long video, as a generalization of detecting `f-formation' in a coffee-break~\cite{Cristani:fformation} and a counterpart to socially non-meaningful collective activities~\cite{Amer:group}.


%The earliest efforts were devoted to analyzing indoor meetings involving handfuls of participants \cite{GaticaPerez,McCowan:meeting}, where descriptors derived from tracking \cite{Smith:track} and pose estimation \cite{Ba:meeting} have been integrated in dynamic hidden Markov models (HMM)~\cite{Zhang:meeting}\todd{need more here. What was the output of their system? How is this different or similar to us? Has any work on meetings defined interaction categories, or recovered social network information from occurrences of these categories?}. Recent efforts have explored more complex visual environments, including the detection and recognition of a small number of pre-defined two-person interaction categories (e.g., hand-shaking, pushing, hugging)~\cite{UTdata}\todd{More here. Cite the related previous work in the CVPR submission.}; the detection of group conversations in a cocktail-party scenario; the detection of certain pre-defined categories of collective large-group behavior~\cite{Choi:context,Choi:recogtrack,Amer:group,Lan:Group}; and PI Li's work on segmenting offensive and defensive players in sports matches~\cite{LiIJCV2012}. The promising results from this assortment of sub-problems is part of what motivates us to pursue a large-scale effort to unify social and visual analysis more broadly, jointly considering all aspects of socialization in image and video data. \todd{It is important that we do not miss any related work in this paragraph. We need to be careful to include everything.}
