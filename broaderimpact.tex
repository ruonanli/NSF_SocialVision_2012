% !TEX root =  SocialVision2012.tex
\vspace{-8pt}
\section{Broader Impacts}
\label{sec:impacts}
\vspace{-8pt}
A framework that enables social understanding from visual data is critical for computer vision systems to meet or exceed the human ability to understand the world from its images. While these ideas have received some attention from the research community, before now we have not had the resources to pursue it in any practical, real-world way. The accelerating development of computational infrastructure, the growth of digital video, and progress in detecting and tracking faces and people has finally made this research possible. 

The proposed  program will seize on this opportunity by concentrating our team's resources on the creation of a  framework for social visual analytics.  We will engage in its creation undergraduate and graduate students, the computer vision research community, and the broader public. These are critical steps toward a future in which machines can better understand and interact with their environments; and they will usher in radical changes to security, personal robotics, augmented reality, human-computer interfaces, security, and video summarization, to name a few.

Our research will also transform computational research in sociology, economics, and education, by providing researchers with access to social patterns that are difficult or impossible to detect with the naked eye. Traditionally, the analysis of interactions from video data has required immense effort by experts. A collections of videos must be viewed repeatedly before salient and meaningful interaction categories can be defined, and their occurrences in new videos must be painstakingly and redundantly coded by trained human observers (e.g.,~\cite{Scherr2009}). Our research will transform machine analyses in these disciplines by greatly simplifying the process of defining interaction categories, by automating the detection of instances of these interactions, and by automatically reconstructing social relationships from these detections. This will enable analyses in a much wider set of environments and over much longer periods of time.

The proposed research program will play an important role in education at both the undergraduate and graduate levels. The fundamental results of this research will be incorporated into the curricula of an existing computer vision class, enhancing senior undergraduate and graduate student education at the participating institution. Undergraduates at the junior and senior levels will be given opportunities for hands-on research experience, including summer research internships that may be funded by future NSF REU supplements. Finally, funding will be used to support two PhD, and these students will be an integral part of both research and educational endeavors in this project. They will participate in all stages, including the design of equipment and algorithms, and presentation of results at major conferences.

Our results will be broadly disseminated in three ways. First, they will be distributed by publications in refereed conferences and journals and by lectures at other institutions.  Second, datasets and source code will be posted online so that they can be easily adopted and extended (see attached Data Management Plan). Third, they will be shared through challenge problem workshops on social visual analytics, held in conjunction with the major  vision conferences, with the small amount of required funding provided by sponsorship agreements. The research team has an established track record in this type of outreach, with PI Zickler having organized tutorials in  vision and  graphics (ICCV 2007, SIGGRAPH 2008) as well as a workshops on computer vision (CRICV 2009, CRICV 2010, CPCV 2011)
