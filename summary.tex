% !TEX root = SocialVision2012.tex
\pagestyle{empty}

\noindent\textsf{RI: Small:}\vspace{0.9ex}\\
\noindent {\bf \textsf{\LARGE Social Visual Analytics}}

\vspace{1.3ex}
\noindent \textsf{\large \em Todd Zickler and Ruonan Li, Harvard University}
\vspace{2.5ex}

\noindent This proposal presents a framework to \emph{see} a social network - one that exploits images and videos and employs computer vision as a new `sensor' to recover social relationships and attributes among individuals in social communities. The mission of computer vision is to extract from visual data useful information about the world, and functioning vision systems are available for many types of information. Yet, there are aspects of the world to which computer vision systems, relative to their human counterparts, remain quite blind. Prominent among these are the \emph{social interactions} and \emph{social relationships} between people. This information is critical for humans as they navigate the world and make decisions.

Contemporary research on social networks, meanwhile, has been revealing us rich semantics by exploiting diverse `conventional social sensors' from textual/voice communications to online message sharing, but it has almost completely ignored visual media from abundant online photos to video volumes harvested by surveillance camera networks. Camera, as a new `community sensor', is much less intrusive than conventional social sensors, but exposes us the `real stories' about the community members from aside and remotely. Facial expressions, gestures, and body poses are critical signals for understanding the interactions and relationships between people, and the only way to get these are from images and videos. 

%Also, it is not just important to understand the type of interaction, but also the context of the scene in which it occurs.

The proposed research will develop foundations for computer vision systems that are `socially aware'. These systems will extract, from images and videos of human gatherings, useful information about the types of social interactions that occur within these gatherings. And by enumerating the different types of interactions that occur over time in large image and video collections, they will extract useful information about the underlying \emph{social network}---the set of social relationships that exist among the observed individuals, groups, and communities. 

\boldstart{Intellectual Merit}. The paradigm of proposed social visual analytics will systematically explores the interactions between visual information and social network, yielding a new foundation for machine discovery of sociological knowledge. The paradigm includes new models and representations for discovering socially-informative visual patterns, especially social interactions, and new computational mechanisms for recovering social networks by exploiting heterogeneous visual cues. The paradigm will in particular account for new challenges arising from both images and societies, and all novel functionalities will build on mature elements in computer vision, such as face/human detection, recognition, tracking, scene analysis, and event/activity interpretation, and will be deployed in the form of software and hardware.

\boldstart{Broader Impacts}. The proposed program will seize on this opportunity for the creation of a framework for social visual analytics. Students, the research community, and the broader public will be engaged, toward a future in which machines can better understand and interact with environments. The proposed program will play an important role in education. The fundamental results will be incorporated into the curricula of an existing computer vision class. Undergraduates will be given opportunities for hands-on research experience including research internships that may be funded by future NSF REU supplements. Funding will be used to support two PhD. Finally, the results will be broadly disseminated by publications, lectures, and online sharing of source code so that they can be adopted and extended, and will be shared through workshops held in conjunction with the major vision conferences. 

\boldstart{Keywords:} Computer vision; social networks; machine learning


%We propose a framework to \emph{see} a social network - one that exploits images and videos and employs computer vision as a new `sensor' by which we aim to recover social relationships and attributes among individuals in our social communities. Contemporary research on social networks has been revealing us rich semantics by exploiting diverse `conventional social sensors' from textual/voice communications to online message sharing, but it has almost completely ignored visual media from abundant online photos to video volumes harvested by surveillance camera networks. Camera, as a new `community sensor', is much less intrusive than conventional social sensors, but exposes us the `real stories' about the community members from aside and remotely: Many of these stories may be too subtle for conventional approaches to grasp but visually sensible. Cameras, equipped with ever-increasing computer vision capabilities, produce socially-informaive data in a quantity thousand times of what conventional sensors do, but these data has largely remained unexplored for social networks. It is now our mission to introduce vision to social networks in this proposed research agenda.
%
%In companion, we propose to understand images and videos by incorporating the contextual knowledge provided by the online social connections where the images and videos are embedded, as well as to distill from images and videos new social semantics previously only under the attention of conventional sociology. This effort builds upon, but will go much beyond the preliminary attempt on recognizing faces and identities under social contexts: Face recognition only tells about \emph{who} are in the visual scene, but we argue that benefits from the socialized nature of media nowadays are not limited to the task of `who'. Knowing more about the social connections among the visual documents helps us to better understand \emph{what} the individuals are doing therein and \emph{where} they geographically are, and even enables us to predict more precisely \emph{whether} a particular event is to happen next.
%
%In fact, both the visual sensing of a social network and the socially assisted understanding of visual materials can benefit from cross-pollination: Visually sensed social ties provide more specific contextual evidences about who are more likely to interact in a new visual scene and what activities they are more likely to be engaged in, while watching the visual co-occurrences and co-activities of two community members may reflect more accurate connections between them that are not easily available from other resources. By socially-aware visual analytics, we mean a comprehensive theoretical and practical infrastructure that we expect to develop during the award period, with the two complementing modules of the visual sensing of a network and the socially assisted visual understanding well integrated.
%
%\boldstart{Intellectual Merit}. We propose a paradigm for socially-aware visual analytics, which systematically explores the interactions between visual information and social communities, yielding a new foundation for machine discovery of sociological knowledge and the potential for new perspectives and applications in visual understanding and computer vision. The paradigm includes new models and representations for networks arising from exploiting visual concepts in images and videos, and includes computational mechanisms to leverage socialized metadata for the new tasks in image and video analysis. The paradigm will, in particular, account for realistic situations in network sensing, such as multi-type overlapping communities and partial noisy observations, which has been largely overlooked by contemporary research, and all novel functionalities will build on mature elements in computer vision, such as face/human detection, recognition, and tracking, scene analysis, and event/activity interpretation, and will be deployed in the form of software and hardware.
%
%\boldstart{Broader Impacts}. The primary application domain of the proposed research is online and networked visual media about humans, while it will be also applicable to broader visual materials involving other agents, such as historical image archives, scientific image collections, biological recordings, and so on. Socialized behaviors are prevalent in social creatures, and our research will shed lights on new approaches to automatic browse, index, and parse them. New insights will be drawn toward diverse disciplines spanning sociology, pedagogy, and statistics, which are still in their infancy in introducing automated approaches exploiting visual information. Besides the tremendous potentials of industrial interests and product implementation that do not need elaboration, the proposed interdisciplinary research will also prompt revolutions in educational programs providing next-generation students with more comprehensive knowledge and broader mastery of skills.
